%%%%%%%%%%%%%%%%%%%%%%%%%%%%%%%%%%%%%%%%%%%%%%%%%%%%%%%%%%%%%%%%%%%%%%%%%%%%%%%%%%%%%%%
%%Universidade Federal de Uberlândia
%%Faculdade de Computação
%%Programa de Pós-graduação em Ciência da Computação
%%Modelo de Plano de Trabalho
%%Proposto por Douglas Farias Cordeiro
%%Baseado no modelo de artigos da SBC, disponível em www.sbc.org.br
%%%%%%%%%%%%%%%%%%%%%%%%%%%%%%%%%%%%%%%%%%%%%%%%%%%%%%%%%%%%%%%%%%%%%%%%%%%%%%%%%%%%%%%


\documentclass[12pt]{article}


\usepackage[brazilian]{babel}
\usepackage{caption}
\usepackage{color}
\usepackage{graphicx,url}
\usepackage{indentfirst}
\usepackage{longtable}
\usepackage{pdflscape}
\usepackage{mathptmx}
\usepackage{scalefnt}
\usepackage{sectsty}
\usepackage{subcaption}
\usepackage[T1]{fontenc}
\usepackage{tabularx}
\usepackage[utf8]{inputenc}

\graphicspath{{images/}}

\sectionfont{\fontsize{12}{12}\selectfont}
\subsectionfont{\fontsize{12}{12}\selectfont}
\pagestyle{empty}
\textwidth 16cm \textheight 23.2cm
\voffset -1.5cm \hoffset -1.4cm

\sloppy

\begin{document}
\begin{center} 
    \bf{{PROJETO DE TESE}\\ $\ $\\
    \textit{(Modelo Aprovado pelo Colegiado em sua 8ª Reunião realizada em 03/09/2015)}\\ $\ $\\}
    {Programa de Pós-graduação em Ciência da Computação\\Universidade Federal de
    Uberlândia}\\
\end{center}

\begin{center}
    \bf{Aluno: \\ $\ $\\
    Orientador: \\ $\ $\\
    Coorientador: \\ $\ $\\
    Data da formalização da coorientação no colegiado: XX/XX/XXXX\\ $\ $\\
    Título do Trabalho: \\ $\ $\\
    Data de Início como Aluno Regular: \\ $\ $\\
    Previsão da Defesa: \\ $\ $\\}
\end{center}

\section{Introdução}
Contextualize o seu trabalho de forma sucinta. Delimite o seu tema de estudo. Convença o leitor da relevância e importância do seu trabalho.

\subsection{Motivação}
Introduza o leitor ao assunto, descreva os fatores motivadores para o desenvolvimento do seu trabalho. Descreva brevemente o estado da arte e indique os problemas que ainda não foram resolvidos. Faça um gancho para a próxima subseção em que você descreve os objetivos do seu trabalho.

\subsection{Objetivos e Desafios da Pesquisa}
Descreva claramente os desafios que o tema propõe e quais os objetivos que se pretende alcançar. Se o tema for muito abrangente, descreva os objetivos em termos de "objetivo geral"e "objetivos específicos". Cuidado com objetivos como "desenvolver um sistema..."; "explorar um método..." Esses objetivos são triviais, ou seja, uma vez desenvolvido o sistema ou explorado o método, independente dos resultados, o objetivo foi atingido. Prefira verbos como: "contribuir", "analisar", "investigar", "comparar". Os membros da banca ao lerem essa seção farão o seguinte questionamento: Algum conhecimento novo para a humanidade foi produzido?

\subsection{Hipótese}
Descreva claramente quais são as hipóteses da sua pesquisa (Uma hipótese é uma suposição para a solução do problema que você pretende desenvolver). Indique quais perguntas estão associadas a sua hipótese. Lembre-se que as hipóteses deverão ser comprovadas via os experimentos propostos na seção que descreve o método de pesquisa.

\subsection{Contribuições}
Liste as contribuições do seu trabalho. 

\section{Revisão da Literatura Correlata}
Descreva os principais trabalhos existentes na literatura da área que estão relacionados com o trabalho que você está propondo e deixe claro qual a sua contribuição em relação a estes trabalhos. O que você está propondo de novo que estes trabalhos não abordaram ou abordaram de forma ineficiente? Eventualmente, se for o caso, introduza nesta seção os conceitos teóricos existentes na literatura que são necessários para a descrição de seu projeto de tese.

\section{Método de Pesquisa}
Descreva de forma mais detalhada sua proposta de trabalho, detalhando as estratégias que pretende utilizar para atingir os objetivos propostos. Descreva o método de pesquisa que deverá ser utilizado para validar a sua hipótese incluindo as medidas de avaliação, conjunto de parâmetros, bases de dados e os trabalhos com os quais a sua proposta será comparada.

\section{Resultados Esperados}
Liste os resultados mais importantes que você pretende obter a partir de seu trabalho de doutorado.

\section{Esquema Geral do Texto da Tese (opcional)}
Descreva o “esqueleto” de sua tese, como vai estruturar os capítulos e seções. Dê um título, mesmo que provisório, a sua tese.

\section{Cronograma de Execução}
Detalhe o cronograma das principais etapas de seu trabalho finalizando pela data da defesa. Para os alunos que planejem fazer doutorado sanduíche é imprescindível destacar quais etapas serão desenvolvidas durante o doutorado sanduíche.

\section{Doutorado Sanduíche (opcional)}
Destaque as razões que tornam a instituição/grupo de pesquisa onde o sanduíche será realizado importante para a realização das etapas do trabalho indicadas na seção anterior.

\bibliographystyle{sbc}
\bibliography{planoDeTrabalho}

\noindent \textbf{Local e Data:}

\noindent \textbf{Assinatura do Orientador:}

\noindent \textbf{Assinatura do Aluno:}
\end{document}