
\documentclass[12pt, a4paper, english, brazil]{article}

% Sistema autor-data com títulos nas referências em negrito
% \usepackage[alf,abnt-emphasize=bf]{abntex2cite}

% \usepackage[alf,abnt-emphasize=bf,abnt-repeated-author-omit=yes,abnt-year-extra-label=yes]{abntex2cite}	% Citações padrão ABNT

% ----------------- link in the PDF
% http://tug.ctan.org/macros/latex/contrib/abntex2/doc/abntex2cite.pdf
%Usando o pacote abntex2cite autonomamente ... é preciso passar a opção carregando o pacote url antes do abntex2cite
\usepackage[hyphens]{url}
\usepackage[bookmarks=false]{hyperref}
\usepackage{hyperref}
% ----------------- link in the PDF

\usepackage[alf,abnt-emphasize=bf,abnt-repeated-author-omit=yes]{abntex2cite}

\usepackage[utf8]{inputenc}	
\usepackage[brazil]{babel}
\usepackage{graphicx,url}
\usepackage{subfigure}
\usepackage{enumitem}
\usepackage{amsfonts}
\usepackage{amsmath}

\usepackage{physics}
\usepackage{comment}

\usepackage{lscape}

\usepackage{fancyhdr}

% -

\usepackage{indentfirst}
% \graphicspath{{img/}}
%\pagestyle{empty}

\usepackage{xcolor}

\newcommand{\textRed}[1]{{{\color{red} #1}}}
\newcommand{\textBlue}[1]{{{\color{blue} #1}}}

\newcommand{\dotsBlue}{\colorbox{orange}{\textcolor{blue}{\dots}}}

\newcommand{\boxRed}[1]{\colorbox{red}{#1}}
\newcommand{\boxYellow}[1]{\colorbox{yellow}{#1}}

\usepackage{tabularx}
\newcolumntype{L}{>{\raggedright\arraybackslash}X}

% -

\textwidth 16cm \textheight 23.2cm
\voffset -1.5cm \hoffset -1.4cm

\sloppy

\begin{document}

\rhead{\thepage}
\pagenumbering{arabic}

\begin{center}
	\bf{\LARGE{PROJETO DE DISSERTAÇÃO}\\ $\ $\\}
	\Large{Programa de Mestrado em Ciência da Computação\\
		Universidade Federal de Uberlândia}\\ $\ $\\
\end{center}

\begin{center}
	\bf{Aluno: João Batista Ribeiro\\ $\ $\\
		Orientador: Prof. Dr. André Ricardo Backes\\ $\ $\\
		Coorientador: Prof. Dr. Maurício Cunha Escarpinati\\ $\ $\\
		Data da formalização da coorientação no colegiado: \colorbox{yellow}{xx/xx/2021}\\ $\ $\\
		Título do Trabalho: \colorbox{yellow}{Detecção de linhas de plantio em plantações de cana-de-açúcar}\\ $\ $\\
		Data de Início como Aluno Regular: março de 2021\\ $\ $\\
		Previsão da Defesa: fevereiro de 2023\\ $\ $\\}
\end{center}

\section{Introdução}
    % => Contextualize o seu trabalho de forma sucinta. Delimite o seu tema de estudo. Convença o leitor da relevância e importância do seu trabalho.

\textBlue{
Falar sobre:
Agricultura de Precisão
    VANTs
Cana-de-açúcar (plantações-área plantada, linhas de plantio, planta cana e planta soca)
desafios (pisoteamento da planta soca, falha no plantio)
linhas de plantio (detecção) 
medição de falhas
Processamento digital de Imagens
}


O rápido crescimento populacional, principalmente do último século, tem impulsionado a demanda por alimentos e a utilização inteligente/sustentável dos recursos naturais. Nesse contexto agricultura aliada tecnologia, chamada de Agricultura de Precisão (AP), busca suprir essa demanda utilizando os recursos sob medida com base nas informações fornecidas pela tecnologia. A AP engloba técnicas de análise espacial da área plantada, informações do solo e das plantas, permitindo aos produtores planejar e monitorar suas plantações \cite{Blasch_2020}.

A Agricultura de Precisão se tornou possível graças ao desenvolvimento e avanços de diferentes tecnologias, como o Sistema de Posicionamento Global (GPS), imagens de satélites, Processamento Digital de Imagens (PDI) e Visão Computacional, sensoriamento remoto entre outras. Possibilitando o desenvolvimento de metologias/técnicas/ferramentas/programas que são aplicados nas várias etapas da agricultura, desde análise e preparação do solo (escassez de determinado nutriente em certa região) até utilização de veículos autônomos para fazer a pulverização de defensivos (com quantidade específica para cada parte do talhão) e na colheita seguindo as linhas de plantio. Muitos dos avanços na AP são fortemente dependentes das tecnologias de processamento digital de imagens \cite{Bolfe_2020}.

As imagens utilizadas na AP têm variadas fontes (e.g., câmeras acopladas em Veículos Aéreos Não Tripulados (VANTs) e satélites) dependendo da aplicação e do \textit{Ground Sample Distance} (GSD) requisitado. GSD refere-se à distância da amostra ao solo, ou seja, quanto cada pixel da imagem obtida representa da regão fotografada. Assim quanto menor for o GSD, mais detalhes a imagem terá da região analisada \cite{Messina_2020}.

As imagens de satélites podem ter custo baixo até elevado (dependendo da qualidade das imagens), mas com baixa temporalidade (imagens de uma mesma área em momentos diferentes), baixo nível de detalhes (maior GSD) em relação aos VANTs e sofrem bastante com as condições climáticas (e.g., nuvens). Por outro lado os VANTs, geralmente têm custo baixo, possibilitam a captura (e recaptura) das imagens assim que necessário (alta temporalidade e disponibilidade), com grande nível de detalhes (pequeno GSD) e não sofrem muito com as nuvens devido a altitude de voo \cite{Candiago_2015, Delavarpour_2021}.

\boxYellow{Definição do problema}

Uma das aplicações importantes da AP é a detecção das linhas de plantio, principalmente porque é utilizado como uma etapa importante para outras aplicações da AP (e.g., detecção de ervas daninhas, mapeamento e previsão de produção de safra, detecção de falhas), além de ser usada por veículos autônomos
\cite{Hassanein_2019}.

\boxYellow{Cana-de-açúcar e o Brasil}

Um dos cenários de utilização da Agricultura de precisão no Brasil é no cultivo de cana-de-açúcar, motivando pesquisadores (e.g., \citeonline{Souza_2017, Souza_2018, Silva_Escarpinati_Backes_2021}) e empresas (e.g., \citeonline{AERO_2017, Sensix_2020, Inforow_2021}) a desenvolverem soluções na área.


O Brasil é um dos maiores produtores de cana-de-açúcar (\cite{Ritchie_2020}) 


\dotsBlue

Na literatura, o trabalho de \cite{Silva_Escarpinati_Backes_2021} contribuiu significativamente para detecção das linhas de plantio, contudo \dotsBlue

Assim, a contribuição do trabalho se dará em
\dotsBlue em imagens de VANTs na AP.


Considerando o cenário que grande utilização de VANTs para obtenção de imagens de Agricultura de precisão, a grande importância da detecção das linhas de plantio, este projeto tem como foco a detecção de linhas de plantio de cana-de-açúcar.

\dotsBlue

\subsection{Motivação}
    % => Introduza o leitor ao assunto, descreva os fatores motivadores para o desenvolvimento do seu trabalho. Descreva brevemente o estado da arte e indique os problemas que ainda não foram resolvidos. Faça um gancho para a próxima subseção em que você descreve os objetivos do seu trabalho.
 
\subsection{Objetivos e Desafios da Pesquisa}
    % => Descreva claramente os desafios que o tema propõe e quais os objetivos que se pretende alcançar. Se o tema for muito abrangente, descreva os objetivos em termos de ``objetivo geral'' e ``objetivos específicos''. Cuidado com objetivos como ``desenvolver um sistema...''; ``explorar um método...'' Esses objetivos são triviais, ou seja, uma vez desenvolvido o sistema ou explorado o método, independente dos resultados, o objetivo foi atingido. Prefira verbos como: ``contribuir'', ``analisar'', ``investigar'', ``comparar''. Os membros da banca ao lerem essa seção farão o seguinte questionamento: Algum conhecimento novo para a humanidade foi produzido?

Este projeto tem como objetivo analisar os métodos de segmentação automática em imagens de VANTs tendo, inicialmente, foco em imagem de plantio de cana-de-açúcar e propor melhorias na detecção de linhas de plantio. 

$=>$ \textRed{talvez} Inicialmente apenas imagem RGB, ou seja, com as 3 bandas, talvez pode ser interessante depois imagem com a banda infravermelho ou infravermelho próximo (dependendo da disponibilidade)

Este projeto tem como objetivos específicos:
\begin{itemize}
    \item Estudar e analisar os métodos propostos na literatura correlata para detecção das linhas de plantio em plantações de cana-de-açúcar, como por exemplo, transformada de Hough e Radon, segmentação (limiarização global, limiarização adaptativa, binarização e método Otsu), redes neurais convolucionais, índices de vegetação, operações morfológicas.
    \item Propor melhorias na detecção das linhas de plantio (com linhas retas e curvas) de imagens obtidas por VANTs nas plantações de cana-de-açúcar com plantas antes do primeiro corte (cana-planta) e depois do primeiro corte (cana-soca)
    \item \dotsBlue
    \item - \textRed{Talvez analisar imagem com banda infravermelho}
    \item Avaliar o método proposto em outros tipos de cultura, como milho, algodão e \dotsBlue
\end{itemize}

% \subsection{Hipótese} 
    % => Descreva claramente quais são as hipóteses da sua pesquisa (Uma hipótese é uma suposição para a solução do problema que você pretende desenvolver). Indique quais perguntas estão associadas a sua hipótese. Lembre-se que as hipóteses deverão ser comprovadas via os experimentos propostos na seção que descreve o método de pesquisa.
% => colocar nos objetivos

\subsection{Contribuições}
    % => Liste as contribuições do seu trabalho. 

\section{Revisão da literatura correlata}
    % => Descreva os principais trabalhos existentes na literatura da área que estão relacionados com o trabalho que você está propondo e deixe claro qual a sua contribuição em relação a estes trabalhos. O que você está propondo de novo que estes trabalhos não abordaram ou abordaram de forma ineficiente? Eventualmente, se for o caso, introduza nesta seção os conceitos teóricos existentes na literatura que são necessários para a descrição de seu projeto de tese.

\textBlue{
Agricultura de Precisão

VANTs (comparar com satélites, GSD, custos e rapidez/temporalidade)

Cana-de-açúcar e o Brasil

PDI (ou está mais para visão computacional?)

    Pré processamento
    Segmentação
    Método de Otsu
    Transformada de Hough
    Transformada de Radon
    Operações Morfológicas
    Índices de Vegetação

Algoritmos Genéticos

Métricas de análise/comparação
    Coeficiente de \textit{Dice}

\textit{Deep Learning}
    CNN
    FCN
    Segmentação semântica (melhor dentro de PDI?)
}

\subsection{Processamento Digital de Imagens}

\subsubsection{Pré-processamento}

\subsubsection{Segmentação}

\subsubsection{Binarização}

\subsubsection{Transformada de Radon}

\subsubsection{Transformada de Hough}

\subsection{Deep Learning}

\subsection{Trabalhos relacionados}

\begin{figure}[htbp]
    \centering
    \includegraphics[width=0.6\textheight]{img/2021_Silva_flow_chart.png}
    \caption{Fluxograma do método desenvolvido por  \citeonline{Silva_Escarpinati_Backes_2021}}
    \label{fig:2021_silva_flow_chart}
\end{figure}

\textRed{lincar imagem ou o texto ou não colocar a imagem}

\citeonline{Silva_Escarpinati_Backes_2021} desenvolveram um método de detecção de linhas de plantio de cana-de-açúcar utilizando algoritmo genético, método de Otsu e transformada de Radon. Para comparar os resultados obtidos foram utilizados 4 \textit{datasets} e suas respectivas linhas de plantio marcadas por um especialista. Os \textit{datasets} utilizados têm canas-de-açúcar em variados estágios (idades), \textRed{contudo} a marcação do especialista foi feita onde existe a linha de plantio e também onde deveria existir, ou seja, locais de falhas no plantio também foram marcados como linha de plantio.

O AG foi utilizado para encontrar uma máscara (\textit{kernel}) (com 27 valores, 9 para cada banda RGB), utilizada como filtro convolucional para combinar as bandas da imagem RGB de entrada em uma imagem de saída em escala de cinza. O AG foi executado por 2700 gerações, com população de 200 indivíduos e taxa de mutação de 0.05 e \textit{crossover} de 0.8. A função de avaliação (\textit{fitness}) do AG busca maximizar o valor de Coeficiente de \textit{Dice} (comparando a imagem gerada ao aplicar a máscara e em seguida o método de Otsu com a marcação feita pelo especialista).

Para analisar os resultados em função do padrão estocástico do AG e sua convergência, foi utilizado a técnica de validação cruzada com 35 imagens divididas em 5 \textit{fodls}. Os autores afirmam que apesar de ter sido obtido máscaras diferentes para cada cenário, elas foram capazes de produzir uma segmentação semelhante em termos de Coeficiente de \textit{Dice}, sendo assim AG foi capaz de calcular uma mascará eficaz independente de conjunto de treinamento utilizado.

O método Otsu foi avaliado na segmentação (nos 4 \textit{datasets}) utilizando o método global e local. No método local utilizaram janelas (\textit{Window}) de $W \times W$ pixels com deslocamento (\textit{Stride}) $S$ e o operador $OR$ para combinar as binarizações locais em uma única imagem binarizada. O método de Otsu global não obteve bons resultados, provavelmente pela grande extensão dos mosaicos (\textit{datasets}) e a variedade de luminosidade presente neles. Por outro lado o método de Otsu local teve melhores resultados, sendo $W=50$ e $S=25$ um dos melhores resultados. As janelas menores permitem o método de capturar as características locais com maior precisão, proporcionado uma melhor binarização e consequentemente um Coeficiente de \textit{Dice} mais próximo de 1.

Para detecção das linhas e refinamento depois da binarização foi utilizado a Transformada de Radon com o método de janelamento (\textit{tiling scheme}), utilizando o mesmo esquema usado no método Otsu, com $W$ e $S$. \dotsBlue

Como resultado geral, a detecção das linhas de plantio ficaram próximas do real (marcação do especialista), lidando melhor com linhas retas e não muito bem com linhas curvas, gerando um efeito serrilhamento (\textit{aliasing}) indesejado. \dotsBlue

\begin{landscape}
\begin{table}
\centering
\caption{Comparação entre trabalhos relacionados}
\label{tab:trab_rela}
\begin{tabularx}{\linewidth}{|L|L|L|L|L|L|} 
\hline
\textbf{Trabalho} & \textbf{Cultura} & \textbf{Principais tecnologias} & \textbf{Resolução espacial} & \textbf{Resultado} & \textbf{Lacunas e Limitações} \\ \hline
\cite{Silva_Escarpinati_Backes_2021} & cana-de-açúcar & Algorítimo Genético, Transformada de Radon, Método de Otsu & 5.3 cm, RGB & Detecção das linhas (retas) de plantio próximas do real & Dependência da parte estocástica do AG, não lida bem com linhas curvas e geração de efeito \textit{aliasing} indesejado \\ \hline
cite trab 2 & cultura & principais tecnologias & resolução espacial & resultado & limitações \\ \hline
cite trab 3 & cultura & principais tecnologias & resolução espacial & resultado & limitações \\ \hline
cite trab 4 & cultura & principais tecnologias & resolução espacial & resultado & limitações \\ \hline
cite trab 5 & cultura & principais tecnologias & resolução espacial & resultado & limitações \\ \hline
\end{tabularx}
\end{table}
\end{landscape}

\section{Método de Pesquisa}
% => Descreva de forma mais detalhada sua proposta de trabalho, detalhando as estratégias que pretende utilizar para atingir os objetivos propostos. Descreva o método de pesquisa que deverá ser utilizado para validar a sua hipótese incluindo as medidas de avaliação, conjunto de parâmetros, bases de dados e os trabalhos com os quais a sua proposta será comparada.

\textBlue{
1 obtenção das imagens

    \textit{dataset} de Silva\_Escarpinati\_Backes\_2021? e/ou outros?

2 análise dos métodos propostos na literatura

3 ...
    pré processamento
    processamento e/ou segmentação
    refinamento - detecção das linhas ou reconstrução
    CNN?

...
}

\section{Resultados esperados}
% => Liste os resultados mais importantes que você pretende obter a partir de seu trabalho de doutorado.

\textBlue{
1 Um método semi-automático ou automático para detecção de linhas de plantio

2 Falhas também?

3 Que consiga lidar bem com linhas retas, linhas curvas, plantações de cana-de-açúcar em vários estágios (idades diferentes) ?

...
}

\section{Esquema Geral do Texto da Dissertação (opcional)}
% => Descreva o “esqueleto” de sua dissertação, como vai estruturar os capítulos e seções. Dê um título, mesmo que provisório, a sua dissertação.

\section{Cronograma de execução}
% => Detalhe o cronograma das principais etapas de seu trabalho finalizando pela data da defesa.

Para cumprir os objetivos descritos, o plano de pesquisa foi dividido em atividades. O cronograma de atividades do projeto pode ser visualizado na \autoref{tab:cronograma}, onde está definida a
duração das principais atividades.

\begin{table}[ht]
\centering
\caption{Cronograma de atividades}
\label{tab:cronograma}
\begin{tabular}{|c|c|c|c|c|c|c|c|c|c|c|c|c|c|c|c|c|}
\hline
          & \multicolumn{2}{c|}{2021} & \multicolumn{12}{c|}{2022}                    & \multicolumn{2}{c|}{2023} \\ \hline
Atividade & N           & D           & J & F & M & A & M & J & J & A & S & O & N & D & J           & F           \\ \hline
A1        & X           & X           & X & X & X & X & X & X & X & X & X &   &   &   &             &             \\ \hline
A2        &             &             & X & X & X & X & X & X & X & X & X & X &   &   &             &             \\ \hline
A3        &             &             &   &   & X & X &   &   &   &   &   &   &   &   &             &             \\ \hline
A4        &             &             &   &   &   & X & X & X &   &   &   &   &   &   &             &             \\ \hline
A5        &             &             &   &   &   &   & X & X &   &   &   &   &   &   &             &             \\ \hline
A6        &             &             &   &   &   &   &   & X & X &   &   &   &   &   &             &             \\ \hline
A7        &             &             &   &   &   &   &   & X & X &   &   &   &   &   &             &             \\ \hline
A8        &             &             &   &   &   &   &   &   & X & X &   &   &   &   &             &             \\ \hline
A9        &             &             &   &   &   &   &   &   &   &   & X & X &   &   &             &             \\ \hline
A10       &             &             &   &   &   &   &   &   &   &   &   & X & X & X &             &             \\ \hline
A11       &             &             &   &   &   &   &   &   &   &   &   &   &   &   & X           & X           \\ \hline
\end{tabular}
\end{table}

As atividades são descritas abaixo:

\begin{itemize}
    \item A1: revisão bibliográfica sobre os métodos de segmentação e detecção de linhas de plantio.
    \item A2: estudo dos algoritmos e métodos encontrados literatura correlata sobre segmentação de imagens.
    \item A3: primeira aplicação e análise dos algoritmos e métodos encontrado em imagens reais.
    \item A4: avaliação dos resultados inciais e propostas de melhorias.
    \item A5: \item A2: estudo dos algoritmos e métodos encontrados literatura correlata sobre \textit{deep learning}.
    \item A6: \dotsBlue
    \item A7: avaliação dos resultados.
    \item A8: readequação de parâmetros e mais experimentos(caso necessário).
    \item A9: elaboração e submissão de artigos científicos para periódicos e/ou congresso da área.
    \item A10: elaboração da dissertação.
    \item A11: defesa da dissertação.
\end{itemize}

% \section{Justificativa pelo atraso na entrega do projeto (tópico obrigatório somente no caso de entrega do projeto foram do prazo regulamentar*)}
% => Descrever as justificativas que levaram ao atraso na entrega do projeto.

% \textbf{*RI-PPGCO/UFU, art. 18, parágrafo único:
% O projeto de Dissertação de Mestrado deverá ser apresentado pelo estudante até o final do segundo semestre letivo, contado a partir da matrícula de ingresso como aluno regular.}

\bigskip
\noindent \boxYellow{Uberlândia, 10 de novembro de 2021.}
% Encaminhar para o e-mail cpgfacom@ufu.br

% \section{Bibliografia}
\bibliography{references}

\end{document}
