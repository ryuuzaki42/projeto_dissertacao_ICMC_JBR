#########################################
# Introdução

\textBlue{
Falar sobre:
Agricultura de Precisão
    VANTs
Cana-de-açúcar (plantações-área plantada, linhas de plantio, planta cana e planta soca)
desafios (pisoteamento da planta soca, falha no plantio)
linhas de plantio (detecção) 
medição de falhas
Processamento digital de Imagens
}


#########################################
# 2 Revisão da literatura correlata

\textBlue{
Agricultura de Precisão

VANTs (comparar com satélites, GSD, custos e rapidez/temporalidade)

Cana-de-açúcar e o Brasil

PDI (ou está mais para visão computacional?)

    Pré-processamento
    Segmentação
    Método de Otsu
    Transformada de Hough
    Transformada de Radon
    Operações Morfológicas
    Índices de Vegetação

Algoritmos Genéticos

Métricas de análise/comparação
    Coeficiente de \textit{Dice}

\textit{Deep Learning}
    CNN
    FCN
    Segmentação semântica (melhor dentro de PDI?)
}

#########################################
# 3 Metodologia de pesquisa

\textBlue{
1 obtenção das imagens

    \textit{dataset} de Silva\_Escarpinati\_Backes\_2021? e/ou outros?

2 análise dos métodos propostos na literatura

3 ...
    pré processamento
    processamento e/ou segmentação
    refinamento - detecção das linhas ou reconstrução
    CNN?

...
}

#########################################
# 4 Resultados esperados

\textBlue{
1 Um método semi-automático ou automático para detecção de linhas de plantio

2 Falhas também?

3 Que consiga lidar bem com linhas retas, linhas curvas, plantações de cana-de-açúcar em vários estágios (idades diferentes) ?

...
}

#########################################
